\documentclass[abbrwidth=6em,tocstyle=ref-genre,toe=false]{ees}

\begin{document}

\eesTitlePage

\def\eesCommentaryAfterToe{%
Bass figures generally only appear in recitatives and chorals (1.1; 1.2; 1.4; 1.6; 2.1; bars 14 and 15 of 2.2; 2.6; bars 1, 2, 6 to 10, and 45 to 48 of 2.8; 2.9; 3.1; bars 1, 2, and 10 of 3.3; 3.5; 3.7; 3.9; 3.11; 4.1; 4.3; 4.5; bars 1 to 3 of 5.2; and 5.4), but not in the remaining movements (1.3, 1.5, 1.7, 2.3 to 2.5, 2.7, 3.2, 3.4, 3.6, 3.8, 3.10, 4.2, 4.4, 4.6, 5.1, 5.3, 5.5, 5.6, and 5.7). The latter bass figures were added by the editor.
}

\eesCriticalReport{}

\eesToc{
Der Dichter dieſes Drama’s hat ſehr weislich die Morgenſtunde gewählt, wo Jeruſalems Einwohner im Tumult ſich dem Schauplatze nähern, wo der damalige Richter, Pontius Pilatus den ihm überantworteten Jeſum richten ſollte. Einen ſo eben zum Oſterfeſte wandernden Fremdling fällt dieſes Hinzuſtrömen des Volkes auf, und wendet ſich fragend an jemand, welches aber gerade einer von denen war,
an welche Jeſus Wunder gethan hatte, und hiermit fängt ſich das Drama an.

\textit{Perſonen des Stücks ſind:}\\[1ex]
Der Fremdling. [S]\\
Der Blindgebohrne. [T]\\
Judas. [B]\\
Petrus. [T]\\
Pilatus. [B]\\
Kaiphas. [B]\\
Joſeph von Arimathia. [A]\\
Nikodemus. [B]\\
Johannes der Liebling Jeſu. [T]\\
Maria die Mutter Jeſu. [S]\\
Maria Magdalena. [A]\\
Ein Prieſter. [B]\\
Der Hauptmann von der römiſchen Schaarwache. [B]\\
Volk, Prieſter, Freunde und Freundinnen Jeſu. [Coro]

\bigskip
\part{act1}

\begin{movement}{derduvoll}
  \voice[Chor]
  Der Du voll Blut und Wunden\\
  für uns am Kreuze ſtarbſt\\
  und unſern letzten Stunden\\
  den größten Troſt erwarbſt,\\
  der Du Dein theures Leben,\\
  noch eh ich war, auch mir\\
  zur Rettung hingegeben,\\
  mein Heil, wie dank ich Dir.
\end{movement}

\begin{movement}{wohineilt}
  \voice[Fremdling]
  Wohin eilt ganz Juda ſo früh?\\
  Wohin ſtrömt die Menge des Volks?

  \voice[Blindgebohrner]
  Weißeſt du nicht die Geſchichte dieſer Nacht?\\
  Kennſt du nicht Jeſum den Propheten Gottes?\\
  Ach! den Frommen! den Gerechten haben unſre Oberſten ergriffen,\\
  und Pilatus übergeben.

  \voice[Fremdling]
  Mir hat auch der Ruf die Weisheit\\
  und die Wunder Jeſu nicht verſchwiegen,\\
  ob ich gleich vom fernen Euphrat komme.\\
  Doch was that er, der Prophet?

  \voice[Blindgebohrner]
  Was er that?\\
  den Kranken gab er Geſundheit,\\
  gab den Lahmen Füße,\\
  Augen den Blinden.
\end{movement}

\begin{movement}{gramlag}
  \voice[Blindgebohrner]
  Gram lag auf mir und Finſterniß.\\
  Er ſchuf mir Sonn und Freuden,\\
  und ſein allmächtig Wort entriß mich\\
  ſchnell der Blindheit Leiden.\\
  Er, dem mein Auge Dank geweint,\\
  iſt ſchuldlos und ein Menſchenfreund.
\end{movement}

\begin{movement}{unddarum}
  \voice[Fremdling]
  Und darum verklagten ihn die Väter?

  \voice[Blindgebohrner]
  Seine Wunder haben ſie beleidigt,\\
  alles Volk folgt dem Propheten,\\
  holet im Triumph ihn ein,\\
  ſinget laut ihm Hoſianna,\\
  ſiehe, das entflammt den Neid,\\
  der die Rache wecket,\\
  daß er ihre übertünchten Laſter,\\
  ihre ſeine Heucheley\\
  oft vor tauſend Zeugen rügte.\\
  Ihrer Wuth gelangs in dieſer Nacht\\
  ihn zu greifen, und zu feßeln.\\
  Mit dem erſten Sonnenlichte\\
  eilten ſie hin zu Pilatus,\\
  klagten ihn des Aufruhrs an.\\
  Jetzt iſt er im Richthaus mit Pilatus.

  \voice[Fremdling]
  Aber hat er Todten auferwecket\\
  und das Auge dir geschaffen –\\
  ſprich, wie kann er leiden?
  \clearpage
  \voice[Blindgebohrner]
  Mir ſchuf er das Auge:\\
  aber dieſe Leiden ſind mir dunkel, Freund, wie dir.\\
  Dort kömmt ſeiner Jünger einer.\\
  Finſter iſt ſein Angeſicht.\\
  Iſt das Judas? Ja er iſts?\\
  Juda! Gieb uns Unterricht,\\
  warum leidet der Prophet?

  \voice[Judas]
  Ha! Er leidet– weil ich ihn verrieth.
\end{movement}

\begin{movement}{werfasst}
  \voice[Judas]
  Wer faßt die Zahl nahmnloſer Quaal,\\
  die auf mein Haupt ſich häuft.\\
  Entflieh, Verbrecher, entflieh dem Rächer,\\
  des Schrecken dich ergreift.\\
  Entflieh der Todten Geiſt, er folget dir,\\
  ſtirb! Todt und Vernichtung mir.
\end{movement}

\begin{movement}{fasstdich}
  \voice[Fremdling]
  Faßt dich Entſetzen?

  \voice[Blindgebohrner]
  Ich verſtumme! doch iſt er ein lauter Zeuge,\\
  daß die Leiden des Propheten unverſchuldet ſind.\\
  Dort reißt ſich aus dem Gedränge\\
  noch ein andrer ſeiner Jünger wehmuthsvoll hervor:\\
  Petrus iſts! Er war bey Jeſu, als er mir die Sonne ſchuf.\\
  Nieder ſinkt ſein Blick; in ſich gekehrt eilt er vorüber. –\\
  Petrus, Petrus höre mich!
\end{movement}

\begin{movement}{lassmich}
  \voice[Petrus]
  Laß mich dieſen bangen Stunden,\\
  laß mich meiner Seele Wunden,\\
  meiner Angſt laß mich entfliehn.\\
  Laß mein Leben mich verweinen,\\
  ich bin nicht mehr von den Seinen,\\
  ach, verläugnet hab ich ihn.
\end{movement}

\part{act2}

\begin{movement}{hilfduihm}
  \voice[Blindgebohrner]
  Hilf du ihm! Gott Abrahams!\\
  Ach, er iſt von allen,\\
  auch von Freunden, die er liebte,\\
  die ſein Leben, ſeine Wunder ſahen,\\
  die ſein Mund gelehrt, verlaßen!

  \voice[Fremdling]
  Alles Volk iſt in Bewegung.

  \voice[Blindgebohrner]
  Ja! Pilatus ſteigt mit Jeſu auf den Richtſtuhl.\\
  Welche Hoheit! welche Ruh im Antlitz Jeſu,\\
  vor ihm ſtehn ergrimmt die Väter.\\
  Welches Urtheil wird der Römer ſprechen!

  \voice[Pilatus]
  Jhr brachtet dieſen Menſchen mir,\\
  und klagtet ihn des Aufruhrs an.\\
  Jch hab ihn ſcharf verhört,\\
  und finde der Sache nicht ihn ſchuldig.\\
  Jſraeliten höret mich.\\
  Zu meiner Linken ſtehet ein Mörder, Barrabas!\\
  zu meiner Rechten euer Jeſus!\\
  Jhr wißt, ich geb an dieſem Feſte\\
  euch der Gefangnen Einen loß.\\
  Welchen ſoll ich euch geben?

  \voice[Kaiphas]
  Jſraeliten!\\
  Dieſen Jeſum habt ihr Abraham läſtern gehört,\\
  Moſis Geſetz entweihen geſehn,\\
  habt gehört, daß er dem Tempel\\
  ſeinen Untergang verkündet:\\
  Rächet an ihm das Geſetz,\\
  und den Tempel Gottes.

  \voice[Pilatus]
  Redet! wen ſoll ich euch geben?\\
  Jeſus oder Barrabas?

  \voice[Chor des Volks]
  Barrabas, Barrabas!

  \voice[Pilatus]
  Was mach ich mit dieſem Jeſu?
\end{movement}

\begin{movement}{kreutzigen}
  \voice[Chor des Volks]
  Kreutzigen laß ihn.

  \voice[Pilatus]
  Er iſt nicht des Todes ſchuldig!

  \voice[Chor des Volks]
  Kreutzigen laß ihn.
\end{movement}

\begin{movement}{ichwasche}
  \voice[Pilatus]
  Jch waſche meine Hände rein,\\
  ich will unſchuldig seyn\\
  am Tode dieſes Frommen!\\
  Es müße des Gerechten Blut\\
  nur dir allein, du Volk voll Wuth,\\
  nur dir zu Schulden kommen.

  \voice[Chor des Volks]
  Über uns komme ſein Blut\\
  und über unſre Kinder.
\end{movement}

\begin{movement}{ochrist}
  \voice[Chor]
  O Chriſt, denk ohne Schauder nie\\
  an dieſe Wuth der Sünder.\\
  Du ſprichſt, die Rache ſtrafte ſie,\\
  und ſtraft noch ihre Kinder.\\
  O Seele, denkſt du auch dabey\\
  an deine Sünden! biſt du frey\\
  von Schuld am Tode Jeſu.
\end{movement}

\begin{movement}{werkann}
  \voice[Chor der Freunde Jeſu]
  Wer kann des Frommen Leiden faßen?\\
  Gott Jacobs, haſt du ihn verlaßen,\\
  den Menſchenfreund und deinen Freund.\\
  Ein leiſer Laut iſt unſre Stimme.\\
  Sie hören nichts in ihrem Grimme,\\
  ſehn nicht das Auge, das ihn weint.\\
  Du Herr erhörſt auch dieſes Flehen.\\
  O eile du, ihm beyzuſtehen.
\end{movement}

\begin{movement}{hoerestdu}
  \voice[Blindgebohrner]
  Höreſt du das Häuflein ſchwacher Freunde?\\
  Ach, ſie klagen den Propheten Gottes!\\
  Aber, welches Ohr iſt nicht verſchloßen\\
  des Mitleids Stimm im Sturm der Leidenſchaft!

  \voice[Fremdling]
  Wo iſt Jeſus?

  \voice[Blindgebohrner]
  Sahſt du nicht, wie Kriegesknechte\\
  ſich um ihn verſammelten?\\
  Jhn hat Pilatus, gleich Verbrechern,\\
  der Geißel übergeben.\\
  Jch irre nicht: dort ſteigt er mit dem Römer\\
  das Tribunal herauf.\\
  Jhr Engel Gottes! welch ein Anblick!
\end{movement}

\begin{movement}{einpurpur}\enlargethispage\baselineskip
  \voice[Blindgebohrner]
  Ein Purpur, eine Dornenkrone,\\
  ihm aufgeſetzt vom bittern Hohne,\\
  Blut rinnt die Schläf herab.

  \voice[Fremdling]
  Ein Rohr trägt er in ſeiner Rechten,\\
  das ſchnöder Spott von niedern Knechten\\
  in ſeine Hand ihm gab.

  \voice[Blindgebohrner]
  Jhr, die ihr keinen Frevel ſcheuet,

  \voice[Fremdling]
  ihr, die ihr euch des Frevels freuet,

  \voice[beyde]
  ihr wißt nicht, wen ihr ſchmäht.

  \voice[Blindgebohrner]
  Jhn rächet einſt der Gott der Götter.

  \voice[Fremdling]
  Er rufet einſt in einem Wetter:

  \voice[beyde]
  Wo iſt, o Juda, dein Prophet?
\end{movement}

\begin{movement}{ihrvaeter}
  \voice[Pilatus]
  Jhr Väter Jſraels ſeht, welch ein Menſch!

  \voice[Chor der\newline Prieſter]
  Kreutzige ihn,\\
  kreutzige ihn!

  \voice[Pilatus]
  Kann nur ſein Tod den heißen Durſt\\
  nach ſeinem Blute ſtillen,\\
  ſo kreutziget ihn ſelber,\\
  er iſt nicht ſchuldig.

  \voice[Chor der\newline Prieſter]
  Läßeſt du dieſen los,\\
  biſt du des Kayſers Freund nicht.\\
  Wer ſich ſelber zum Könige macht,\\
  der iſt wider den Kayser.

  \voice[Pilatus]
  Jhr wollts, ihr wollts! ihr Wüthenden!\\
  ſo werd er denn gekreutziget,\\
  gekreutzigt euer König.
\end{movement}

\begin{movement}{sehtwelch}
  \voice[Chor]
  Seht! welch ein Menſch, ach, ſeht!\\
  Schmerzhafte Dornen krönen\\
  ſein majeſtätiſch Haupt,\\
  doch mag die Welt dich höhnen.\\
  Mein Jeſus, mir bleibſt du\\
  ein König auch noch hier.\\
  Voll Ehrerbietigkeit\\
  beug ich die Knie vor dir.

  Seht! welch ein Menſch! er muß\\
  vom Rohrſtab frech zerſchlagen\\
  noch in der eignen Hand\\
  des Frevlers Werkzeug tragen.\\
  Ach wißt, daß dieſe Hand\\
  ein eiſern Zepter trägt.\\
  Sorgt, Frevler, daß ſie euch\\
  nicht einſt im Zorn zerſchlägt.

  Seht! welch ein Menſch! mein Herz\\
  im Leibe will mir brechen,\\
  ob dieſe Leiden ſchon\\
  mir ewgen Preis verſprechen.\\
  Herr! ich kan ihrer nie\\
  mich ohne Wehmuth freun.\\
  Ach, laß mich ja für dich\\
  nie Spott und Schande ſcheun.
\end{movement}

\part{act3}

\begin{movement}{siefuehren}
  \voice[Blindgebohrner]
  Sie führen! ach! ſie führen ihn zum Tode!\\
  Er ſoll ſterben, er, den ich ſagen hörte:\\
  „Einſt kommt die Stunde, in welcher alle,\\
  die in den ſtillen Gräbern ſchlafen,\\
  des Menſchenſohnes Stimme hören\\
  und gehn hervor.“
\end{movement}

\begin{movement}{jesuschristus}
  \voice[Blindgebohrner]
  Jeſus Chriſtus wird das Leben\\
  allen Todten wiedergeben,\\
  und der Staub ſoll auferſtehn.\\
  Auferſtehn, nicht mehr zu ſterben,\\
  und des Vaters Reich zu erben,\\
  mit ihm hin zum Thron zu gehn.\\
  Den kann Todes Nacht nicht decken,\\
  der die Todten wird erwecken.
\end{movement}

\begin{movement}{dustaerkest}
  \voice[Fremdling]
  Du ſtärkeſt mich! Er wird nicht ſterben.\\
  Laß dem Leidenden uns folgen.\\
  Höreſt du nicht Klageton, leis und ſeitwärts!\\
  Noch ſind Edle, die ihn klagen!\\
  Laß den Klagenden uns von ferne nahn!
\end{movement}

\begin{movement}{sieher}
  \voice[Joseph]
  Sieh! er träget ſein Kreutz,\\
  ach, auf dem blutenden Rücken,\\
  kraftlos ſchwanket er nun,\\
  er erlieget der Laſt.

  \voice[Nikodemus]
  Dennoch ſchaut er umher\\
  voll Ruhe der Seelen.\\
  Mitleid redet ſein Blick\\
  zu den Mördern umher.

  \voice[Joseph]
  Ach, du göttlicher Mann,\\
  wirſt du dem Schwachen verzeihen,\\
  daß in der Sünder Gericht\\
  er nicht laut dich bekannt?

  \voice[Nikodemus]
  Ach, du Gottes Prophet!\\
  den ich im Stillen beſuchte,\\
  in dem Schatten der Nacht:\\
  Wirſt du dem Schwachen verzeihn?
\end{movement}

\begin{movement}{seydmir}
  \voice[Blindgebohrner]
  Seyd, ach ſeyd mir geſegnet,\\
  ſtille Freunde des Propheten!\\
  Jch kenn euch wohl,\\
  ihr willigtet in ihren Blut[rauſch] nicht,\\
  nicht in den Rath, mich in den Bann zu thun,\\
  als ich den laut bekannte,\\
  der mir das Auge ſchuf.

  \voice[Nikodemus]
  Biſt du hier? du, der mit Muthe\\
  unſre Aelt’ſten ſtraft’ und ſagte:\\
  „Wäre dieſer nicht von Gott,\\
  nimmer könnt’ er Blinden Augen geben.“\\
  Ach, viel muthiger warſst du,\\
  wareſt ſeliger als ich.

  \voice[Joseph]
  Hörſt du, Nikodemus, wie ihn Zions Töchter klagen?\\
  Auch ſie muthiger als wir!\\
  Mitten unter ſeinen Mördern ſteigt ihr Lied empor.
\end{movement}

\begin{movement}{grossist}
  \voice[Chor der Töchter Zions]
  Gros iſt ſeine Quaal,\\
  blutig ſein Geſicht,\\
  wie die Roſ’ im Thal,\\
  die der Sturmwind bricht,\\
  ſinkt der Edle hin!\\
  Schwestern, weinet ihn,\\
  weinet, weinet ihn!
\end{movement}

\clearpage
\begin{movement}{jesuswendet}
  \voice[Blindgebohrner]
  Jeſus wendet ſich hin zu denen, die ihn klagen.\\
  Möcht ich von den Leidenden einen Laut noch hören!\\
  Eilen will ich durch die Haufen,\\
  einen Laut von ſeinen Lippen,\\
  eh er ſtirbt, zu hören.

  \voice[Joseph]
  Wie er eilt! muthig iſt der Redliche!\\
  ſterben würd’ er, ſterben,\\
  könnt er den Propheten retten.\\
  Blindgeb.Jch vernahms, was er geſprochen, Freunde!\\
  Schreckensvolle Worte:
\end{movement}

\begin{movement}{weintnicht}
  \voice[Blindgebohrner]
  „Weint nicht um mich!\\
  weint über euch!\\
  Es nahen ſich angſtvolle Tage von fern,\\
  Gewittern Gottes gleich.\\
  Dann hört man dieſe Jammerklage:\\
  Heil der, die nicht gebohren hat.\\
  Heil der! die nicht geſäuget hat!\\
  Dann rufen ſie mit Todesſchrecken:\\
  Fallt auf uns, Berg’, uns zu bedecken.“
\end{movement}

\begin{movement}{odassich}
  \voice[Nikodemus]
  O, daß ich dieſe Tage des Jammers nicht erlebe!\\
  daß dieſes Auge breche,\\
  eh dieſer Schauplatz voller Greul ſich öffnet!\\
  Sein Blut, ach, des Gerechten Blut\\
  wird Gottes Rache ſchnell beflügeln.

  \voice[Joseph]
  O Nikodemus!\\
  es erlieget der Ermüdete der Laſt, er ſinkt nieder.\\
  Welch Getümmel! ach! ſie zwingen, die Blutgiergen,\\
  einen Wandrer, ihm ſein Kreutz zu tragen.
\end{movement}

\begin{movement}{erhoeredieses}
  \voice[Joseph, Blindgebohrner, Nikodemus]
  Erhöre dieſes heiße Flehen,\\
  laß ihn, du des Gerechten Gott,\\
  den langſam bangen Tod nicht ſehen,\\
  den Tod am Kreutz, den Sclaventod.\\
  Verkürz ihm auf einmal\\
  die ſchrecklichſten der Leiden,\\
  daß ſich an ſeiner Quaal\\
  nicht ſeine Mörder weiden.
\end{movement}

\begin{movement}{einopfer}
  \voice[Chor]
  Ein Opfer nach dem ewgen Rath, belegt mit unſern Plagen,\\
  um deines Volkes Mißethat gemartert und geſchlagen,\\
  gehſt du den Weg zum Kreutzesſtamm\\
  in Unſchuld ſtumm, gleich als ein Lamm,\\
  das man zur Schlachtbank führet.\\
  Freywillig, als der Helden Held trägſt du aus Liebe für die Welt\\
  den Tod, der uns gebühret.
\end{movement}

\part{act4}

\begin{movement}{heiligheilig}
  \voice[Nikodemus]
  Heilig, heilig, heilig biſt du Gott!\\
  aber unerforſchlich auch.\\
  Bluten ſoll er! der Gerechte,\\
  ſterben unter Mißethätern!\\
  Wende dich von dieſen Tiefen, Seele,\\
  die du ſchwindelſt,\\
  wende dich weg und bete ſchweigend an.

  \voice[Joseph]
  Laß uns am Fuß des Hügels weilen,\\
  ich kan nicht ſehn des Frommen Sterben.\\
  Sie nahn ſich ihm – die Kreutziger!\\
  O weh! ſie heften ihn ans Kreutz.
\end{movement}

\begin{movement}{raecherschau}
  \voice[Fremdling, Joseph, Blindgebohrner, Nikodemus]
  Rächer ſchau vom Himmel nieder,\\
  all ihr Engel ſchauet nieder:\\
  Es fließt ſein Blut.\\
  Klaget all in lauten Chören,\\
  daß es Höhn und Tiefen hören:\\
  Es fließt ſein Blut.
\end{movement}

\begin{movement}{emporgerichtet}
  \voice[Joseph]
  Empor gerichtet iſt das Kreutz!\\
  und der blicket hernieder!\\
  Nikodemus! dieſe Blicke ſtärken dich.\\
  Ach, ſanfte Liebe! und Verzeihung reden ſie.

  \voice[Blindgebohrner]
  Sieh, Johannes der Geliebte dränget\\
  durch die dichten Haufen gegen uns ſich her.\\
  Hier, Johannes, ſtehen Freunde\\
  deines Meiſters des Gerechten.\\
  Wareſt du dem Kreutze nahe?\\
  Sahſt du ihn leiden?

  \voice[Johannes]
  Nahe war ich, ſah ihn leiden.\\
  Habet ihr den Blick geſehen,\\
  den er auf die Mörder warf?

  \voice[Joseph]
  Wir bemerkten dieſen Blick.

  \voice[Johannes]
  Aber, was der Mund der Liebe ſagte,\\
  habt ihr nicht gehört:\\
  Kindlein! Nichts als Huld und Liebe\\
  quillet aus der Seele Jeſu:\\
  Herab ſieht er: „Vergieb, o Vater, ihnen,\\
  ſie wißen nicht, was ſie thun.“

  \voice[Freunde Jeſu]
  Liebe, Liebe, Gottes Liebe!

  \voice[Johannes]
  Nikodemus, ſaheſt du\\
  ſeine ſchmerzensvolle Mutter?\\
  Jeſu Wink hat mir geboten,\\
  ſie zu ſeinem Kreutz zu führen.

  \voice[Maria Magdalena]
  Johannes!

  \voice[Johannes]
  Jüngerin von Magdala! du hier?\\
  die Mutter des Propheten,\\
  ſie wird nicht ferne ſeyn.

  \voice[Maria Magdalena]
  Sie ſtehet ſtumm und thränenlos\\
  nicht fern von hier,\\
  ein Schwerdt geht ihr\\
  durch ihre Seele.

  \voice[Johannes]
  Maria! Jeſu Wink geboth\\
  zu ſeinem Kreutze dich zu führen.

  \voice[Maria]
  Ach ſtärke mich, Gott Jſraels!\\
  Jch folge dir, Johannes.

  \voice[Ein Prieſter]
  Ja folg ihm, ſieh ihn bluten!\\
  Todesbläße ſteiget auf die Wangen des Empörers,\\
  glaubt ihr noch an ihn?
\end{movement}

\begin{movement}{anderenhalf}
  \voice[Chor der\newline Prieſter]
  Andern half er,\\
  kan ſich ſelber nicht helfen.\\
  Jſt er Chriſtus,\\
  o ſo ſteig er vom Kreutze,\\
  und wir alle gläuben an ihn.\\
  Jſt er Gottes Sohn,\\
  er ſteige nieder vom Kreutze,\\
  er, der Gottes Tempel zerbricht\\
  und in dreyen Tagen ihn bauet.\\
  Gott hat er vertraut,\\
  der mag ihn erlöſen.
\end{movement}

\begin{movement}{gelobtsey}
  \voice[Nikodemus]
  Gelobt sey Gott! die Wüthenden,\\
  ſie wenden ſich von uns hinweg.\\
  Johannes führet ſie zurück,\\
  des Propheten Mutter!\\
  Ach Maria! welchen Troſt gab er dir, der Göttliche?

  \voice[Johannes]
  Liebend neigte der Prophet\\
  gegen ſie ſein Angeſicht.\\
  „Meine Mutter!“, rief er nieder,\\
  „dieſer iſt dein Sohn!“\\
  und zu mir: „Johannes! dieſe, deine Mutter.“
\end{movement}

\begin{movement}{owelche}
  \voice[Maria]
  O welche Wonne, welch Entzücken\\
  ſich in mein leidend Herz ergoß!

  \voice[Johannes]
  O welcher Troſt mit ſeinen Blicken\\
  vom Kreutz in meine Seele ſchoß!

  \voice[Maria Magd.]
  Kein Auge weint zu ihm vergebens.

  \voice[Maria]
  Sein Wort

  \voice[Johannes]
  sein Blut

  \voice[alle]
  iſt voll des ewgen Lebens.

  \voice[Maria]
  Jm Tode noch voll Zärtlichkeit

  \voice[Maria Magd.]
  gab er uns dieſe Seeligkeit

  \voice[alle]
  o preiſet ihn mit uns, ihr Frommen.

  \voice[Maria]
  Nun freut mein Geiſt ſich wieder ſein.

  \voice[Maria Magd., Johannes]
  Er ſprach: „Sey nicht mehr Todes Pein,

  \voice[alle]
  auf uns iſt Gottes Ruh gekommen.“
\end{movement}

\part{act5}

\begin{movement}{schwarzegrauenvolle}
  \voice[Chor des Volks und der Prieſter]
  Schwarze grauenvolle Wolken\\
  ſchweben über uns daher.\\
  Höret! wie ſie furchtbar rauſchen,\\
  höret, welch Getöſe in den Tiefen.\\
  Wehe, weh uns!\\
  Es zerreißet laut die Erde, wir verſinken.\\
  Höret ihr den Felſen krachen,\\
  er zerſpringt, zerſchmettert uns.\\
  Wie der Sturmwind ſchrecklich brauſet,\\
  welch ein Donner, welch ein Schlag!\\
  Gottes Rache, Zaubereyen,\\
  ſchone unſrer, Gott der Götter,\\
  wir, wir tödteten ihn nicht.\\
  Kaiphas, du riefſt den Donner: Rette du uns.\\
  Laßt uns fliehen, wir verſinken.\\
  Gottes Rache! laßt uns fliehen.
\end{movement}

\begin{movement}{gottesschrecken}
  \voice[Nikodemus]
  Gottes Schrecken ſchlagen ſie,\\
  über ſie kommt ſchon ſein Blut.

  \voice[Johannes]
  Du, der du ſie mit Schrecken ſchlägeſt,\\
  ſtärke deines Sohnes Freunde,\\
  denn er ſtirbt am Kreutze.\\
  Schaut hinauf zu ſeinem Kreutze:\\
  Tiefer ſinkt ſein Haupt zum Herzen,\\
  bleicher ſind die Wangen,\\
  die gebrochnen Augen hebet er gen Himmel –\\
  ich vergehe!

  \voice[Nikodemus]
  Ach, er rufet, betet, neiget ſein Haupt.

  \voice[Chor]
  O wehe, wehe! er ſtirbt, er ſtirbt.
\end{movement}
\enlargethispage\baselineskip
\begin{movement}{ihraugen}
  \voice[Chor]
  Jhr Augen weint! der Menſchenfreund\\
  ließ dort für uns ſich ſchlagen.\\
  Jeſu, unſre Mißethat\\
  wirkte deine Plagen.

  Erbarme dich, erbarme dich!\\
  Herr, unſer, wenn wir ſterben.\\
  Laß auf dich entſchlafen uns\\
  und dein Reich ererben.
\end{movement}

\begin{movement}{erderam}
  \voice[Hauptmann]
  Er, der am Kreutze ſtarb,\\
  er war ein heiliger, gerechter,\\
  frommer Mann, ihr Römer! das war er!
\end{movement}

\begin{movement}{sahetihr}
  \voice[Hauptmann]
  Sahet ihr den göttlich leiden,\\
  ſo litt nie ein Sterblicher!\\
  Hörtet ihr ihn laut verſcheiden,\\
  ſo ſtarb kein Gekreuzigter!\\
  Fühltet ihr die Erde beben?\\
  Gott, ſein Gott bezeuget ſchon,\\
  er lebt’ eines Frommen Leben,\\
  er war wahrlich Gottes Sohn!

  \voice[Chor]
  Wahrlich, er war wahrlich Gottes Sohn!
\end{movement}

\begin{movement}{weinetihn}
  \voice[Chor der Freundinnen Jeſu]
  Weinet ihn, bange trauervolle Lieder!\\
  O wie ſank ſein Haupt hernieder,\\
  wie hat ihn der Tod entſtellt?

  \voice[Chor der Freunde Jeſu]
  Singet ihm!\\
  Bange leidensvolle Stunden ſind vorüber,\\
  überwunden hat der Göttliche, der Held.

  \voice[Chor]
  Halleluja, halleluja!\\
  Gott heißt ſeine Feinde ſchweigen,\\
  Erd und Sonn und Himmel zeugen:\\
  Er iſt heilig und ſein Sohn.
\end{movement}

\begin{movement}{unsersuenden}
  \voice[Chor]
  Unſer Sünden Angſt zu lindern,\\
  o Jeſu! rede zu uns Sündern\\
  vom Seegens vollen Golgatha!\\
  Ach laß Gott nicht mit uns reden,\\
  es ſpricht dein Blut: Nun Heil uns Blöden,\\
  Gott iſt mit Gnad uns wieder nah.\\
  Für uns zum Fluch gemacht\\
  riefſt du: Es iſt vollbracht!\\
  Jauchzet, jauchzet,\\
  es iſt vollbracht,\\
  nun ganz vollbracht,\\
  der ſtarb hat alles wohl gemacht.
\end{movement}
}

\eesScore

\end{document}
